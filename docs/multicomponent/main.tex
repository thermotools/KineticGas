\documentclass{article}
\usepackage[T1]{fontenc}		% Nødvendig for fonter
\usepackage[utf8]{inputenc}	% Nødvendig for æøå
\usepackage{babel}		% Fornorsker dokumentet
\usepackage{graphicx}	% For å kunne inkludere grafikk

\usepackage{siunitx}	% For SI-enheter
\DeclareSIUnit{\bar}{bar}
\DeclareSIUnit{\atm}{atm}
\DeclareSIUnit{\celcius}{\degree C}
\DeclareSIUnit{\molar}{M}

\usepackage[version=4]{mhchem}	% For kjemi
\usepackage{enumitem}	% For tilpasning av lister
\usepackage{fullpage}	% For mindre marger (=bedre plassutnyttelse)
\usepackage{tabularx}
\usepackage{multirow}	% Tillater tabellceller som går over flere rader/kolonner
\usepackage{amsmath,amssymb}
\usepackage{fancyhdr}
\usepackage{comment}
\usepackage{xcolor}
\usepackage[many]{tcolorbox}
\usepackage{courier}
\usepackage{booktabs}
\usepackage[hidelinks]{hyperref}
\usepackage{listings}
\usepackage{todonotes}

\usepackage{subfig}

\setlength{\parindent}{0pt}
\newcommand{\pfrac}[2]{\left(\frac{#1}{#2}\right)}
\newcommand{\pder}[2]{\frac{\partial #1}{\partial #2}}
\newcommand{\ppder}[2]{\pfrac{\partial #1}{\partial #2}}
\renewcommand{\Vec}[1]{\boldsymbol{\mathrm{#1}}} % pretty vectors
\renewcommand{\vec}[1]{\Vec{#1}}
\newcommand{\Mat}[1]{\underline{\pmb{#1}}}
\newcommand{\diag}{\mathrm{diag}}

\title{Documentation for multicomponent solutions}
\author{Vegard Gjeldvik Jervell}

\begin{document}

\maketitle

\tableofcontents

\section{Introduction}

The starting point for developing a kinetic model for multicomponent, high density mixtures is the same as that for the binary, single component case, with a minor modification. The Boltzmann equations for an $N$ component mixture may be written as

\begin{equation}
    \lsp \pder{}{t} + \vu_i \cdot \nabla + \pfrac{\Vec{F}_i}{m_i} \cdot \pder{}{\vu_i} \rsp f_i = \sum_j J_{ij}(f_i f_j), \hspace{.5cm} i = \{1, 2, ..., N\}
    \label{eq:boltzmann_multi}
\end{equation}

where $t$ is the time, $\vu_i$ is the velocity, $\Vec{F}_i$ is the sum of external forces, $m_i$ is the mass and $f_i$ is the velocity distribution function (vdf.) of species $i$. $J_{ij}$ is the streaming operator which becomes
\begin{equation}
    J_{ij}(f_i f_j) \equiv \int \int \int \chi_{ij}(\Vec{r}, \Vec{r} + \sigma_{ij}\hat{k}) f_i'(\Vec{r})f_j'(\Vec{r} + \sigma_{ij}\hat{k}) - \chi_{ij}(\Vec{r}, \Vec{r} - \sigma_{ij} \hat{k})f_i(\Vec{r})f_j(\Vec{r} - \sigma_{ij}\hat{k}) b\d b \d \epsilon \d \vu_j
    \label{eq:streaming_op}
\end{equation}

where $\hat{k}$ is the unit vector connecting the two particles, $b$ is the impact parameter and $\epsilon$ is the angular coordinate in the plane of $b$. The prime in $f_i'$ denotes functions of the post-collision velocities. In the same manner as for low-density mixtures, the streaming operator describes the rate of change in the vdf. of species $i$ due to collisions with species $j$. 

The modification when comparing to the low-density streaming operator is the introduction of the factor $\chi_{ij}$, the pair distribution function of the particles, which modifies the probability of finding particles $i$ and $j$ at positions $\Vec{r}_i$ and $\Vec{r}_j$. Furthermore, the vdf. of particle $j$ in the integral of Equation \eqref{eq:streaming_op} is evaluated at $\Vec{r} \pm \sigma_{ij} \hat{k}$ rather than at $\Vec{r}$. Here, $\sigma_{ij}$ is taken to be the distance between the centre of mass of the particles ''at contact''. For hard spheres, this definition is unproblematic but for particles interacting with some realistic potential the definition of being ''at contact'' is slightly less clear. For now, $\sigma_{ij}$ may be regarded as a parameter in the range of the molecular sizes, that is independent of particle velocity and position.

Following the Enskog solution method,\cite{cohen_1} de Haro et al. find that a first approximation to the vdf. may be written as
\begin{equation}
    f_i^{(1)} = f_i^{(0)}\lsp 1 + \Phi_i \rsp
\end{equation}

where 

\begin{equation}
    f_i^{(0)} = n_i \pfrac{m_i}{2 \pi k_B T}^{\frac{3}{2}} \exp \lsp - \sU_i^2 \rsp
    \label{eq:eq_vdf}
\end{equation}

is the Maxwell distribution function, with the peculiar velocity $\vU_i \equiv \vu_i - \vu^m$ defined relative to the \textit{centre of mass} velocity $\vu^m$ and the dimensionless peculiar velocity defined as $\sU_i^2 \equiv \frac{m_i}{2 k_B T}U_i$. $n_i$ is used to denote the particle density of species $i$.

Equivalently to the low-density case, $f_i^{(0)}$ satisfies the conservation equations of mass, energy and momentum exactly. That is,
\begin{equation}
    \begin{split}
        \int f_i^{(0)} \d \vu_i &= n_i, \hspace{.25cm} \forall \hspace{.25cm} i\\
        \sum_i \int f_i^{(0)} m_i \vu_i \d \vu_i &= \rho \vu^m \\
        \sum_i \int f_i^{(0)} \frac{m_i}{2} \vU_i ^2 \d \vu_i &= \frac{3}{2} n k_B T
    \end{split}
\end{equation}

where $\rho$ denotes the mass density of the mixture. Thus, for all $r > 0$ we can require that
\begin{equation}
    \begin{split}
        \int f_i^{(r)} \d \vu_i &= 0, \hspace{.25cm} \forall \hspace{.25cm} i\\
        \sum_i \int f_i^{(r)} m_i \vu_i \d \vu_i &= 0 \\
        \sum_i \int f_i^{(r)} \frac{m_i}{2} \vU_i ^2 \d \vu_i &= 0.
    \end{split}
\end{equation}

The equation of conservation of momentum is obtained by multiplying Equation \eqref{eq:boltzmann_multi} by $m_i \vu_i$. Reordering this equation and inserting for $f_i^{(0)}$, one can identify the hydrostatic pressure as

\begin{equation}
    p = p^k + p^\phi, \hspace{.5cm} p^k = n k_B T, \hspace{.5cm} p^\phi = \frac{2 \pi}{3} n^2 k_B T \sum_i \sum_j x_i x_j \sigma_{ij}^3 \chi_{ij}
\end{equation}

where $x_i$ denotes the mole fraction of species $i$.

In determining the first order approximation to the vdf. it is found that $\Phi_i$ is of the form
\begin{equation}
    \Phi_i = - \frac{1}{n} \vA_i \nabla \ln T - \frac{1}{n} \mB_i : \nabla \vu^m + \frac{1}{n} H_i \nabla \cdot \vu^m - \frac{1}{n} \sum_j \vD_i^{(j)} \vd_j'
\end{equation}

where $\vd_j'$ is defined by

\begin{equation}
    \vd_i = \sum_{j \neq i} \omega_j \vd_i' - \omega_i \vd_j'
\end{equation}

with $\omega_i$ denoting the weight fraction of species $i$ and
\begin{equation}
    \vd_i = - \frac{\rho_i}{\rho n k_B T} \lsp \nabla p + \sum_j \rho_j \lrp \frac{\Vec{F}_i}{m_i} - \frac{\Vec{F}_j}{m_j}\rrp \rsp + \sum_j x_i \lrp \delta_{i,j} + \frac{4 \pi}{3} n_j M_{ij} \sigma_{ij}^3 \chi_{ij}\rrp \nabla \ln T + \frac{x_i}{k_B T} \nabla_T \mu_j
    \label{eq:diffusion_driving_force}
\end{equation}

where $\delta_{i,j}$ is the Kronecker delta and $M_{ij} = \frac{m_i}{m_i + m_j}$. The final term, the gradient in chemical potential at constant temperature may be rewritten as

\begin{equation}
    \frac{x_i}{k_B T} \nabla_T \mu_i = \frac{x_i}{k_B T} \sum_j \ppder{\mu_i}{n_j}_{T,n_{k \neq j}} \nabla n_j
\end{equation}

yielding
\begin{equation}
    \vd_i = - \frac{\rho_i}{\rho n k_B T} \lsp \nabla p + \sum_j \rho_j \lrp \frac{\Vec{F}_i}{m_i} - \frac{\Vec{F}_j}{m_j}\rrp \rsp + \sum_j x_i \lrp \delta_{i,j} + \frac{4 \pi}{3} n_j M_{ij} \sigma_{ij}^3
    \chi_{ij}\rrp \nabla \ln T + \frac{1}{n} E_{ij} \nabla n_j
    \label{eq:diffusion_driving_force2}
\end{equation}

where one should recall that $n_i$ denotes the particle \textit{density} of component $i$.

The response functions $\vA_i$, $\mB_i$, $H_i$ and $\vD_i^{(j)}$ are related to the thermal conductivity, shear viscosity, bulk viscosity and diffusion coefficient of the mixture. In the same manner as for a dilute mixture, one may determine the transport coefficients by writing the response functions as polynomial expansions in the Sonine polynomials, and requiring that these expansions obey the constrains posed by the summational invariants.

In the following sections the resulting equations for the transport coefficients, and their relation to the fluxes will be given. In the case of diffusion, the matter of how the diffusion coefficient should be defined will be addressed.

\section{Diffusion}

The molar flux of species $i$ in the centre-of-mass frame of reference is related to the vdf. as
\begin{equation}
    \mflux_i = n_i (\bvu_i - \vu^m) = \int f_i \vU_i \d \vu_i.
    \label{eq:molar_flux_mass_for}
\end{equation}
The diffusive response functions $\vD_i^{(j)}$ are written as the Sonine polynomial expansions
\begin{equation}
    \vD_i^{(j)} = \frac{m_i}{2 k_B T} \sum_{p = 0}^\infty d_{i, j}^{(p)} S_{3/2}^{(p)}(\sU^2)
\end{equation}
At constant temperature, the integral of Equation \eqref{eq:molar_flux_mass_for} may be evaluated in terms of the $d_{i,j}^{(r)}$ expansion coefficients as
\begin{equation}
    \mflux_i = - \frac{x_i}{2} \sum_j d_{i, j}^{(0)} \Vec{d}_j
\end{equation}
where $\Vec{d}_j$ may be appropriately simplified in the absence of a temperature gradient.

This section will first describe the equations that must be solved to determine the expansion coefficients of the diffusive response function, then the relationship between the diffusive fluxes and driving forces is discussed. Finally, several ways of defining the diffusion coefficient in a mixture are introduced, and explicit expressions relating the diffusion coefficients to the expansion coefficients are given.

\subsection{Determining the expansion coefficients}
\label{sec:diffusion_exp_coeff}

Using the orthogonality properties of the Sonine polynomials and inserting this expansion into the constrains posed by the summational invariants, one finds that the expansion coefficients must satisfy 
\begin{equation}
    \begin{split}
        \sum_{j=1}^s \sum_{q = 0}^{N} \Lambda_{ij}^{pq}d_{j,k}^{(q)} &= \frac{8}{25 k_B}\left(\delta_{i,k} - \frac{\rho_i}{\rho}\right)\delta_{p,0}, \hspace{1cm}
        \begin{cases}
            i = \{2, 3, ..., s\}&\\
            p = \{0, 1, ..., N\}& \\
            k = \{1, 2, ..., s\}&
        \end{cases}
        \\
        \sum_{j=1}^s \sum_{q=0}^N \Lambda_{1j}^{(pq)} d_{j,k}^{(q)} &= 0, \hspace{1cm}
        \begin{cases}
            k = \{1, 2, ..., s\}&\\
            p = \{1, 2, ..., s\}&
        \end{cases}
        \\
        \sum_{j = 1}^s \frac{\rho_j}{\rho} d_{j,k}^{(0)} &= 0, \hspace{1cm} k = \{1, 2, ..., s\}
    \end{split}
    \label{eq:diffusion_eq_set}
\end{equation}

where $\Lambda_{ij}^{pq}$ are given by

\begin{equation}
    \Lambda_{ij}^{pq} = x_i x_j \chi_{ij} \lsp S_{3/2}^{(p)}(\sU_i^2)\vsU_i^2, S_{3/2}^{(p)}(\sU_j^2)\vsU_j^{2}\rsp_{ij} + \delta_{i, j} x_i \sum_{k} x_{k} \chi_{ik} \lsp S_{3/2}^{(p)}(\sU_i^2)\vsU_i^2, S_{3/2}^{(p)}(\sU_i^2)\vsU_i^{2}\rsp_{ik}.
    \label{eq:lambda_ijpq}
\end{equation}

The square bracket integrals may be written as linear combinations of the collision integrals.\cite{kinetics_1} Thompson et al. identify these integrals as

\begin{equation}
    \begin{split}
        \lsp S_{3/2}^{(p)}(\sU_i^2)\vsU_i^2, S_{3/2}^{(p)}(\sU_j^2)\vsU_j^{2}\rsp_{ij} &= 8 M_{ij}^{q + \frac{1}{2}} M_{ji}^{p + \frac{1}{2}} \sum_{\ell = 1}^{\min [p, q] + 1} \sum_{r = \ell}^{p + q + 2 - \ell} A_{pqr \ell} \Omega_{ij}^{(\ell)}(r) \\
        \lsp S_{3/2}^{(p)}(\sU_i^2)\vsU_i^2, S_{3/2}^{(p)}(\sU_i^2)\vsU_i^{2}\rsp_{ik} &= 8 \sum_{\ell = 1}^{\min [p, q] + 1} \sum_{r = \ell}^{p + q + 2 - \ell} A_{pqr \ell}' \Omega_{ik}^{(\ell)}(r).
    \end{split}
\end{equation}

The set of equations \eqref{eq:diffusion_eq_set} may be written as

\begin{equation}
    \Mat{D} \Vec{d} = \Vec{\delta}
    \label{eq:diffusion_eq_matr}
\end{equation}
where $\Mat{D}$ is a ($Ns^2 \times Ns^2$) matrix consisting of the blocks
\begin{equation}
    \Mat{D} = 
    \begin{bmatrix}
        \Mat{\omega} \\ \Mat{\Lambda}_1^{(p>0)} \\ \Mat{\Lambda}_{i>1}
    \end{bmatrix}
\end{equation}

with
\begin{equation}
    \begin{split}
        \Mat{\omega} &= 
        \begin{bmatrix}
            \Vec{\omega}^{(1)} \\ \Vec{\omega}^{(2)} \\ \vdots \\ \Vec{\omega}^{(s)}
        \end{bmatrix}\\
        \Vec{\omega}^{(k)} &= 
        \begin{pmatrix}
            0 & \hdots & \times Ns(k - 1) & \hdots & 0 & \omega_1 & \omega_2 & \hdots & \omega_s & 0 & \hdots & \times s(Ns - N(k - 1) - 1) & \hdots & 0
        \end{pmatrix}
    \end{split}
\end{equation}

\begin{equation}
    \begin{split}
        \Mat{\Lambda}_1^{(p>0)} &= 
        \begin{bmatrix}
            \Vec{\Lambda}(1, 1) \\ \Vec{\Lambda}(1, 2) \\ \vdots \\ \Vec{\Lambda}(1, N) \\ \Vec{\Lambda}(2, 1) \\ \vdots \\ \Vec{\Lambda}(s, N)
        \end{bmatrix},\\
        \Vec{\Lambda}(k, p) &= 
        \begin{pmatrix}
            0 & \hdots & \times Ns(k - 1) & \hdots & 0 & \Lambda_{11}^{p0} & \Lambda_{12}^{p0} & \hdots & \Lambda_{1s}^{p0} & & \Lambda_{11}^{p1} & \hdots & \Lambda_{1s}^{pN} & 0 & \hdots & \times Ns(s - k) & \hdots & 0\\
        \end{pmatrix}
    \end{split}
\end{equation}

\begin{equation}
    \begin{split}
        \Mat{\Lambda}_{i>1} &= 
        \begin{bmatrix}
            \Mat{\Lambda}_{k=1} & & & \\
            & \Mat{\Lambda}_{k=2} & & \\
            & & \ddots & \\
            & & & \Mat{\Lambda}_{k=s}    
        \end{bmatrix}
        \\
        \Mat{\Lambda}_{k} = 
        \begin{bmatrix}
            \Mat{\Lambda}_{i>1}^{(00)} & \Mat{\Lambda}_{i>1}^{(01)} & \hdots & \Mat{\Lambda}_{i>1}^{(0N)} \\
            \Mat{\Lambda}_{i>1}^{(10)} & \ddots             &        & \vdots \\
            \vdots             &                    & \ddots & \vdots \\
            \Mat{\Lambda}_{i>1}^{(N0)} & \hdots & \hdots  & \Mat{\Lambda}_{i>1}^{(NN)}
        \end{bmatrix}&, \hspace{.5cm}
        \Mat{\Lambda}_{i>0}^{(pq)} = 
        \begin{bmatrix}
            \Lambda_{21}^{pq} & \Lambda_{22}^{pq} & \hdots & \hdots & \Lambda_{2s}^{pq} \\
            \Lambda_{31}^{pq} & \ddots & & & \vdots \\
            \vdots & & \ddots & & \vdots \\
            \Lambda_{s1}^{pq} & \hdots & \hdots & \Lambda_{s,(s-1)}^{pq} & \Lambda_{ss}^{pq}
        \end{bmatrix}
    \end{split}
\end{equation}

$\Vec{d}$ and $\Vec{\delta}$ are given as

\begin{equation}
    \Vec{d} = 
    \begin{pmatrix}
        d_{1, 1}^{(0)} \\ d_{2, 1}^{(0)} \\ \vdots \\ d_{s, 1}^{(0)} \\ d_{1, 1}^{(1)} \\ \vdots \\ d_{s, 1}^{(1)} \\ \vdots \\ d_{s, 1}^{(N)} \\ d_{1, 2}^{(0)} \\ \vdots \\ d_{s, 2}^{(N)} \\ \vdots \\ d_{s, s}^{(N)}
    \end{pmatrix}
    , \hspace{1cm}
    \Vec{\delta} = - \frac{8}{25 k_B}
    \begin{pmatrix}
        0 \\ \vdots \\ \times Ns \\ \vdots \\ 0 \\ \Vec{\delta}^{(1)} \\ \Vec{\delta}^{(2)} \\ \vdots \\ \Vec{\delta}^{(s)}
    \end{pmatrix}, \hspace{1cm}
    \Vec{\delta}^{(k)} = 
    \begin{pmatrix}
        \omega_2 \\ \vdots \\ \omega_{k-1} \\ \omega_k - 1 \\ \omega_{k+1} \\ \vdots \\ \omega_s \\ 0 \\ \vdots \\ \times (N-1)(s-1) \\ \vdots \\ 0
    \end{pmatrix}
\end{equation}

\subsection{Flux-Force relations}

Inserting the polynomial expansion of $\vD_i^{(j)}$ into Equation \eqref{eq:molar_flux_mass_for} one finds that at uniform temperature and pressure, and in the absence of external forces
\begin{equation}
    \begin{split}
        \mflux_i &= - \frac{x_i}{2} \sum_j d_{i,j}^{(0)} \Vec{d}_j\\
        &= - \frac{x_i}{2n} \sum_j d_{i,j}^{(0)} \sum_k E_{jk} \nabla n_k \\
        &\equiv - \frac{x_i}{2n}\sum_k \nabla n_k \sum_j E_{jk} d_{i,j}^{(0)}
    \end{split}
    \label{eq:diffusive_flux}
\end{equation}

where $E_{ij} \equiv \frac{n_i}{k_B T} \ppder{\mu_i}{n_j}_{T,n_{k \neq j}}$.

At this point it should be noted that the driving forces $\Vec{d}_j$ are not independent, but satisfy the relation $\sum_j \Vec{d}_j = 0$, following from the Gibbs-Duhem equation. Due to this dependency there are several manners in which one may identify the diffusion coefficient of a multicomponent mixture. Equation \eqref{eq:diffusive_flux} will be the starting point from which the diffusion coefficient is related to the polynomial expansion coefficients. It is therefore convenient to denote

\begin{equation}
    \mflux_i = - \sum_j D^{(E,m)}_{ij} \nabla n_j
    \label{eq:diffusive_flux2}
\end{equation}
where 
\begin{equation}
    D^{(E,m)}_{ij} \equiv \frac{x_i}{2n} \sum_k E_{kj} d_{i,k}^{(0)}
\end{equation}

In order to identify the diffusion coefficient related to different driving forces, it may be useful to express $\mflux_i$ in terms of the mole fraction gradients $\nabla x_i$, as these conveniently sum to zero. Taking

\begin{equation}
    \begin{split}
        \nabla n_k &= n \nabla x_k + x_k \nabla n\\
        &= n\nabla x_k + x_k \sum_\ell \ppder{n}{x_\ell}_{T,p} \nabla x_\ell \\
        &= n \nabla x_k + x_k \sum_\ell - \frac{\ppder{p}{x_\ell}_{T,n}}{\ppder{p}{n}_{T,\vx}} \nabla x_\ell \\
        &= n \nabla x_k + x_k \sum_\ell - \frac{\ppder{T}{x_\ell}_{p,n}}{\ppder{T}{n}_{p,\vx}} \nabla x_\ell \\
        &= \sum_{\ell} \lsp n \delta_{k,\ell} - x_k \frac{\ppder{p}{x_\ell}_{T,n}}{\ppder{p}{n}_{T,\vx}} \rsp \nabla x_\ell \\
        &\equiv \sum_{\ell} \vartheta_{k\ell} \nabla x_\ell
    \end{split}
    \label{eq:nabla_k_nabla_x}
\end{equation}

where the final equality defines $\vartheta_{k \ell}$ and the equality on the fourth line can be derived by differentiating the total differential of density

\begin{equation}
    \d n = \ppder{n}{T}_{p, \vx} \dT + \ppder{n}{p}_{T, \vx} \dT + \sum_i \ppder{n}{x}_{T, p} \d x_i
\end{equation}

with respect to $x_i$ and $n$, and using the fact that $\ppder{n}{x_i}_{p, n} = \ppder{n}{x_i}_{T, n} = 0$ and $\ppder{n}{n}_{p, \vx} = \ppder{n}{n}_{T, \vx} = 1$. Inserting Equation \eqref{eq:nabla_k_nabla_x} into Equation \eqref{eq:diffusive_flux} yields
\begin{equation}
    \begin{split}
        \mflux_i &= - \frac{x_i}{2n}\sum_k \sum_{\ell} \vartheta_{k \ell} \nabla x_\ell \sum_j E_{jk} d_{ij}^{(0)}\\
        &= - \frac{x_i}{2n} \sum_\ell \nabla x_\ell \sum_{k} \vartheta_{k \ell} \sum_{j} E_{jk} d_{ij}^{(0)}.
    \end{split}
\end{equation}

Alternatively, one may choose to express the flux using a set of independent gradients. Using the condition $\sum_j \Vec{d}_j = 0$ to express $\nabla n_i$ as 

\begin{equation}
    \begin{split}
        \sum_j \Vec{d}_j = \sum_j \sum_k E_{jk}\nabla n_k &= 0\\
        \nabla n_i \sum_j E_{ji} + \sum_{k \neq i} \nabla n_k \sum_j E_{jk} &= 0\\
        \nabla n_i = - \sum_{k \neq i} \nabla n_k \frac{\sum_j E_{jk}}{\sum_j E_{ji}} \equiv - \sum_{k \neq i} \nabla n_k \frac{E'_k}{E'_i},
    \end{split}
    \label{eq:molar_grad_relation}
\end{equation}

where the final equality defines $E'_i = \sum_{j} E_{ji}$, and inserting into Equation \eqref{eq:diffusive_flux} one arrives at

\begin{equation}
    \begin{split}
        \mflux_i &= - \frac{x_i}{2n}\sum_k \nabla n_k \sum_j E_{jk} d_{ij}^{(0)} \\
        &= - \frac{x_i}{2n} \lrp \nabla n_i \sum_j E_{ji} d_{ij}^{(0)} + \sum_{k \neq i} \nabla n_k \sum_j E_{jk} d_{ij}^{(0)} \rrp \\
        &= - \frac{x_i}{2n}\sum_{k \neq i} \nabla n_k \sum_j (E_{jk} - E_{ji} \frac{E'_k}{E'_i}) d_{ij}^{(0)}.
    \end{split}
    \label{eq:diff_fluxforce_3}
\end{equation}

As shown elsewhere,\cite{prosjektoppgave} the molar fluxes in the centre of mass FoR can be translated to another frame of reference $B$ via the matrix transformation

\begin{equation}
    \Vec{J}^{(B)} = \Mat{\Psi}^{B,m} \Vec{J}^{(m)}
    \label{eq:flux_transform_for}
\end{equation}
where 
\begin{equation}
    \psi_{ij}^{Bm} = \delta_{i,j} - x_i \left( \frac{b_j}{b} - \frac{M_j b_\ell}{b M_\ell}\right)
\end{equation}

where $\delta_{i,j}$ is the Kronecker delta, $x_i$ denotes the mole fraction of species $i$, $M_j$ is the molar mass of species $j$, $b = \ppder{B}{n}_{T,p,\Vec{x}}$ is the molar value of the extensive property $B$, $b_j \equiv \ppder{B}{n_j}_{T,p,n_{k \neq j}}$ is the partial molar value of $B$ with respect to species $j$ and $\ell$ denotes an arbitrary component.

Note that because $\ell$ denotes an arbitrary component

The vector $\Vec{J}^{(B)}$ is
\begin{equation}
    J^{(B)} = 
    \begin{pmatrix}
        J^B_1 \\ J^B_2 \\ \vdots \\ J^B_s
    \end{pmatrix}
\end{equation}
and equivalent for $\Vec{J}^{(m)}$. 

This transformation can also be used to translate the ''apparent'' diffusion coefficients between different frames of reference, as is discussed in the following section.

\subsection{Defining the diffusion coefficients}
The diffusion coefficients of a mixture may be defined through the relationship between the fluxes and driving forces in a mixture. In this section Equation \eqref{eq:diffusive_flux2}, describing the relationship between the molar flux in the centre of mass FoR and the molar density gradients, will be related to other flux-force relationships through which the diffusion coefficients are commonly defined.

\subsubsection{Binary mixtures}

The molar flux in a binary mixture is often described by Ficks law in the centre-of-moles frame of reference. This is convenient in the case of fairly dilute mixtures in which equimolar counter diffusion applies. Ficks law for a binary mixture reads
\begin{equation}
    \flux{1}{n,n} = - D_{12}^{\text{Fick}}\nabla n_1
\end{equation}
where the superscript $(n, n)$ indicates that the flux is in the molar basis, in the centre-of-moles FoR. Translating the fluxes of Equation \eqref{eq:diffusive_flux2} to the centre of moles frame of reference yields
\begin{equation}
    \begin{pmatrix} \flux{1}{n,n} \\ \flux{1}{n,n} \end{pmatrix}
    =
    - \Mat{\Psi}^{(n,m)}\Mat{D}^{(E,m)} \begin{pmatrix} \nabla n_1 \\ \nabla n_2 \end{pmatrix}
    \equiv
    - \Mat{D}^{(E,n)} \begin{pmatrix} \nabla n_1 \\ \nabla n_2 \end{pmatrix}.
\end{equation}

where the final equality defines $\Mat{D}^{(E,n)} \equiv \Mat{\Psi}^{(n,m)}\Mat{D}^{(E,m)}$. By Equation \eqref{eq:molar_grad_relation}
\begin{equation}
    \nabla n_2 = - \frac{E'_1}{E'_2} \nabla n_1 
\end{equation}
such that 
\begin{equation}
    \flux{1}{n,n} = - D_{11}^{(E,n)} \nabla n_1 - D_{12}^{(E,n)} \nabla n_2 = \lrp D_{12}^{(E,n)} \frac{E'_1}{E'_2} - D_{11}^{(E,n)} \rrp \nabla n_1.
\end{equation}
Thereby, the Fickean diffusion coefficient in the molar basis is identified as
\begin{equation}
    D_{12}^{\text{Fick}} = D_{12}^{(E,n)} \frac{E'_1}{E'_2} - D_{11}^{(E,n)}.
\end{equation}

\subsubsection{Multicomponent mixtures}

de Haro et al. define the diffusion coefficient through a multicomponent generalization of Ficks law in the centre of mass FoR
\begin{equation}
    \mflux_i = \frac{1}{m_i} \sum_{j \neq i} D_{ij}^{fick} m_j \nabla n_j.
    \label{eq:multicomp_fick}
\end{equation}
This expression has the advantage of reducing to the commonly used expression 
\begin{equation}
    \mflux_i = - D_{12} \nabla n_1
\end{equation}

in the case of a binary mixture. Comparing Equation \eqref{eq:multicomp_fick} to \eqref{eq:diff_fluxforce_3} one finds that
\begin{equation}
    D_{ij}^{fick} = - \frac{x_i m_i }{2n m_j} \sum_j (E_{jk} - E_{ji} \frac{E'_k}{E'_i}) d_{ij}^{(0)}.
    \label{eq:fick_diffusion_coeff}
\end{equation}

An alternative to the generalized Fickean diffusion coefficient is the Maxwell-Stefan diffusion coefficient for multicomponent mixtures. It is defined by
\begin{equation}
    \nabla x_i = - \sum_{j \neq i} \frac{x_i x_j}{D_{ij}^{M.S.}} (\bvu_i - \bvu_j)
\end{equation}

As shown by 

For a realistic mixture, it is of interest to separate the factors $E_{ij}$, which may be computed from an equation of state (EoS) from the polynomial expansion coefficients $d_{i,j}^{(0)}$ which can be computed from kinetic theory without requiring an accurate equation of state. Writing the fluxes as

\begin{equation}
    \mflux = \Mat{D}^{kin} \Mat{\Gamma} \nabla \vn
\end{equation}

and comparing to Equation \eqref{eq:diffusive_flux}, one finds that this separation can be achieved by using

\begin{equation}
    \Mat{D}^{kin} = - \frac{1}{2n}
    \begin{bmatrix}
        x_1 d_{1,1}^{(0)} & x_2 d_{1,2}^{(0)} & \hdots & x_1 d_{1,s}^{(0)} \\
        x_1 d_{2,1}^{(0)} & x_2 d_{2,2}^{(0)} & & \vdots \\
        \vdots & \vdots & \ddots & \vdots \\
        x_1 d_{s,1}^{(0)} & x_2 d_{s,2}^{(0)} & \hdots & x_s d_{s,s}^{(0)}
    \end{bmatrix}
    , \hspace{.5cm}
    \Mat{\Gamma} = 
    \begin{bmatrix}
        E_{11} & E_{12} & \hdots & E_{1s} \\
        E_{21} & E_{22} & & \vdots \\
        \vdots & \vdots & \ddots & \vdots \\
        E_{s1} & E_{s2} & \hdots & E_{ss}
    \end{bmatrix}
\end{equation}

One should note that these matrices are not invertible and that the coefficients do not reduce to the Fickean diffusion coefficients in the case of a binary mixture. This formulation is still convenient, as it allows one to transform the coefficients directly to another frame of reference $B$ by the transformation

\begin{equation}
    \begin{split}
        \flux{}{(B)} &= \Mat{\Psi}^{B, m} \mflux \\
        &= \Mat{\Psi}^{B, m} \Mat{D}^{kin} \Mat{\Gamma} \nabla \vn \\
        &= \Mat{D}^{kin, B} \Mat{\Gamma} \nabla \vn \\
        \Mat{D}^{kin, B} &\equiv \Mat{\Psi}^{B, m} \Mat{D}^{kin}
    \end{split}
\end{equation}

where $\Mat{D}^{kin, B}$ is the apparent kinetic diffusion coefficient matrix in the $B$ FoR. This may then be transformed as desired by the same procedure as that used to arrive at Equation \eqref{eq:fick_diffusion_coeff} of one wishes to express the fluxes as a function of only independent gradients.
\section{Thermal Diffusion}

Thermal diffusion coefficients are defined through an extension of the equations in Sec. \ref{sec:diffusion}, using the same notation as is present there, thermal diffusion coefficients computed using the KineticGas package are defined through
\begin{equation}
    \begin{pmatrix}J_1 \\ J_2 \\ \vdots \\ J_N \end{pmatrix}^{(n, f)} = 
    \begin{pmatrix}
    D_{T,1} \\ D_{T,2} \\ \vdots \\ D_{T,N}    
    \end{pmatrix}^{(f, l)} \nabla \ln T -
    \begin{bmatrix}
    D_{11} & D_{12} & \hdots & D_{1N} \\
    D_{21} & D_{22} & \hdots & D_{2N} \\
    \vdots & \vdots & \ddots & \vdots \\
    D_{N1} & D_{N2} & \hdots & D_{NN}
    \end{bmatrix}^{(f, l)}
    \begin{pmatrix}\nabla c_1 \\ \nabla c_2 \\ \vdots \\ \nabla c_N \end{pmatrix}
    \label{eq:thdiff_definition}
\end{equation}
or, more compactly
\begin{equation}
    \Vec{J}^{(n, f)} = \Vec{D}_T^{(f, l)} \nabla \ln T - \Mat{D}^{(f, l)} \nabla \Vec{c}.
\end{equation}

Note that because in the presence of a temperature gradient, the Gibbs-Duhem equation no longer reduces to 
\begin{equation}
    \sum_i x_i \nabla \mu_i = 0,
\end{equation}
the choice of dependent component ($l$) will not only effect the diffusion matrix $\Mat{D}^{(f, l)}$, but also the thermal diffusion vector $\Vec{D}_T^{(f, l)}$. Just as for the diffusion matrix, the frame of reference and choice of dependent component for thermal diffusion coefficients is selected with the options \code{frame\_of\_reference} and \code{dependent\_idx}, with the \code{thermal\_diffusion\_coeff} method. 

Also, just as for the diffusion matrix, thermal diffusion coefficients computed using the KineticGas package are defined through Eq. \eqref{eq:thdiff_definition}, i.e. with $\nabla \ln T$ and the \textit{molar concentration gradients} as the driving forces, and with the fluxes on a \textit{molar} basis.
\section{Thermal Conductivity}

The heat flux in the centre of mass frame of reference is related to the vdf. as
\begin{equation}
    \qflux = \sum_i \int f_i \frac{m_i}{2}\vU_i^2 d\vu_i
\end{equation}

Having obtained expressions for the thermal- and diffusive response functions, $\vA_i$ and $\vD_i^{(j)}$ in the previous sections this integral may be evaluated to yield

\begin{equation}
    \begin{split}
        \qflux &= - \frac{5 k_B T}{4n} \sum_i K_i x_i \lrp a_i^{(1)} \nabla \ln T - \sum_j d_{i,j}^{(1)} \vd_j\rrp \\
        &\hspace{1cm} - \frac{4 k_B T}{3} \sum_i \sum_j \pfrac{2\pi m_i m_j kB T}{m_i + m_j}^{\frac{1}{2}} \frac{n_i n_j \sigma_{ij}^4 \chi_{ij}}{m_i + m_j} \nabla \ln T\\
        &\hspace{1cm} + k_B T \sum_i \sum_j \frac{2 \pi}{3} n_j \sigma_{ij}^3 (M_{ij} - M_{ji})\chi_{ij}\mflux_i\\
        &\hspace{1cm} + \frac{5 k_B T}{2} \sum_i \lrp 1 + \sum_j\frac{2 \pi}{3} n_i \sigma_{ij}^3\chi_{ij}\rrp \frac{m_i}{m_j} \mflux_i.
    \end{split}
    \label{eq:heat_flux_general}
\end{equation}

In the absence of a mass flux, Fouriers law applies and we have
\begin{equation}
    \qflux = - \lambda \nabla T
    \label{eq:fouriers_law}
\end{equation}

where $\lambda$ is the conductivity. When all mass fluxes vanish in the presence of a temperature gradient, the molar density gradients in $\vd_i$ may be replaced by the thermal diffusion ratios, such that

\begin{equation}
    \begin{split}
        \vd_i &= \sum_j x_i \lrp \delta_{i,j} + b_{ij} - k_{T,i} E_{ji} \rrp  \nabla \ln T, \hspace{1cm} \mflux_k = 0 \hspace{.5cm} \forall \hspace{.5cm} k\\
        &\equiv \sum_j d_j^{th} \nabla \ln T,
    \end{split} 
\end{equation}
where the second equality defines $d_j^{th}$. Furthermore, because all mass fluxes have vanished, these $d_j^{th}$ must satisfy
\begin{equation}
    \sum_j d_{i,j}^{(0)} d_j^{th} = a_i^{(0)} 
\end{equation}

as is seen by setting the left hand side of Equation \eqref{eq:molar_flux_with_temp} to zero. Comparing Equations \eqref{eq:heat_flux_general} and \eqref{eq:fouriers_law} we identify the thermal conductivity as

\begin{equation}
    \begin{split}
        \lambda &= - \frac{5 k_B T}{4n} \sum_i K_i x_i \lrp a_i^{(1)} - \sum_j d_{i,j}^{(1)} d_j^{th}\rrp \\
        &\hspace{1cm} - \frac{4 k_B T}{3} \sum_i \sum_j \pfrac{2\pi m_i m_j k_B T}{m_i + m_j}^{\frac{1}{2}} \frac{n_i n_j \sigma_{ij}^4 \chi_{ij}}{m_i + m_j}
    \end{split}
\end{equation}
\section{Viscosity}
The flux of momentum $\Mat{P}$ is related to the velocity distribution function as 
\begin{equation}
    \Mat{P} = \sum_i \int f_i m_i \vU_i \d \vu_i.
    \label{eq:momentum_flux}
\end{equation}

The hydrodynamic response functions $\Mat{B}_i$ and $H_i$ may be written as the polynomial expansions

\begin{equation}
    B_i = \frac{m_i}{2 k_B T} \sum_{r = 0}^\infty b_i^{(r)} S_{5/2}^{(r)}(\vsU^2), \hspace{1cm} H_i = \sum_{r = 0}^{\infty} h_r^{(0)} S_{1/2}^{(r)}(\vsU^2)
\end{equation}
where $B_i$ is defined by $\Mat{B}_i \equiv B_i\lrp \vU_i \vU_i - \frac{1}{3}U_i^2 \Mat{I}\rrp$. Inserting these expansions into Equation \eqref{eq:momentum_flux}, and applying the conservation law for momentum yields a set of equations for the expansion coefficients as
\begin{equation}
    \sum_{j = 1}^s \sum_{q = 0}^N B_{ij}^{(pq)} b_j^{(q)} = \frac{2}{k_B T}x_i K_i' \delta_{p,0}, \hspace{1cm}
    \begin{cases}
        i = \{1, 2, ..., s\} &\\
        p = \{0, 1, ..., N\} &
    \end{cases}
\end{equation}

where 

\begin{equation}
    \begin{split}
        K_i' &= 1 + \frac{8 \pi n}{15} \sum_j x_j M_{ji} \sigma_{ij}^3 \chi_{ij}\\
        B_{ij}^{(pq)} &= \frac{2}{5 k_B T} \left\{ x_i x_j \lsp S_{5/2}^{(p)}(\vsU_i^2)\overset{\circ}{\vsU_i \vsU_i}, S_{5/2}^{(q)}(\vsU_j^2)\overset{\circ}{\vsU_j \vsU_j} \rsp_{ij}\right. \\
        &\hspace{2cm} \left.+ \delta_{i,j} \sum_k x_i x_k \lsp S_{5/2}^{(p)}(\vsU_i^2)\overset{\circ}{\vsU_i \vsU_i}, S_{5/2}^{(q)}(\vsU_i^2)\overset{\circ}{\vsU_i \vsU_i} \rsp_{ik}\right\}
    \end{split}
\end{equation}

with 
\begin{equation}
    \overset{\circ}{\vsU_i \vsU_i} \equiv \vsU \vsU - \frac{1}{3}\sU^2 \Mat{I}.
\end{equation}

These bracket integrals are exactly the ones identified by Thompson et al. as linear combinations of the collision integrals.\cite{kinetics_viscosity}

\subsection{Determining the expansion coefficients}

\subsection{Flux force relations}

\subsection{Viscosity in terms of the expansion coefficients}

Comparing Equations ... and ... one can identify the shear viscosity as
\begin{equation}
    \eta = \frac{k_B T}{2} \sum_i K_i' x_i b_i^{(0)} + \frac{4}{15} \sqrt{2 \pi k_B T} \sum_i \sum_j \sqrt{\frac{m_i m_j}{m_i + m_j}} n_i n_j \sigma_{ij}^4 \chi_{ij}
\end{equation}
\section{Contact diameters}

The ''contact diameter'', $\sigma_{ij}$ has been mentioned several times thus far, and has been taken to be some distance in the range of the particle sizes. In the case of additive hard spheres, the contact diameter can unambiguously be defined as the distance between the centre of mass of the two particles at contact. However, for Mie particles this definition is not equally straight forward. In this section, various ways of defining the contact diameter, and the inherent underlying assumptions behind the different definitions will be discussed.

Firstly, it is worth mentioning that applying the multicomponent, density corrected solutions proposed by de Haro et al. to Mie fluids implies the assumption that the contact diameters are independent of particle velocities at collision. This assumption is necessary due to the fact that the integral of Equation \eqref{eq:streaming_op} is one over the velocity space. The contact diameters are permitted to be functions of the temperature, and thereby the mean velocities, as well as density and composition but must be constant for all particles in a given state.

One could, in principle, define the contact diameters as some function of the velocities, but this would severely limit the possibility of utilising previously obtained results from the literature. Therefore, such an approach has not been attempted here.

When defining the contact diameters of Mie particles, we note the two roles this distance plays. The first is describing the covolume of the mixture, and the modified probability of finding two particles at contact through the radial distribution function ''at contact''. The second is describing the instantaneous transfer of energy and momentum from one particle to the other when particles collide. This effect manifests itself as the second terms in the expressions for the conductivity and viscosity, which depends on the density, contact diameters and rdf. but not on the polynomial expansion coefficients. Because the contact diameter plays two distinctly different roles, it is not necessarily so that the length one should use in these two roles must be the same.

To evaluate the radial distribution function ''at contact'' a highly convenient choice of the contact diameter is the Mie parameter $\sigma_{ij}$. This allows one to directly apply the expressions proposed by Lafitte et al. for the rdf. at contact.\cite{lafitte2013accurate} This formulation of the rdf. at contact has been shown to give accurate predictions of thermodynamic properties of fluids, and the associated distance is therefore likely to give a good representation of the covolume of the mixture. Therefore, it is believed that using $\sigma_{ij}$ as the contact diameter when computing the rdf. at contact is not only a convenient choice, but also a choice that allows accurate representation of the modified probability of contact between particles due to volume exclusion.

Regarding the second property described by the contact diameter, the instantaneous transfer of energy and momentum at the moment of collision, a distance more directly related to the collision dynamics was chosen. Firstly, note that the equilibrium vdf. given in Equation \eqref{eq:eq_vdf} does not depend on the contact diameters. We regard a colliding pair of particles, and define the contact diameter as the average distance of closest approach ($R$) during collision where the particles repel each other (e.g. collisions where $\theta < \frac{\pi}{2}$). Further, we compute this average for a mixture at equilibrium, when $f_i = f_i^{(0)}$. The contact diameter is then given by

\begin{equation}
    \Bar{R}_{ij} = \int_{0}^{\infty} \int_0^{b'} R_{ij}(g_{ij}, b) \d b \d g_{ij}
    \label{eq:R_defl}
\end{equation}

where $g_{ij}$ is the relative speed of the colliding pair and $b'$ is the solution to the equation

\begin{equation}
    \theta_{ij}(b'; g_{ij}) = 0.
\end{equation}

This integral is somewhat computationally expensive to evaluate, but may be simplified by noting that $R(g_{ij}; b)$ is reasonably symmetric about $g = \Bar{g}$, the average relative speed, as shown in Figure \ref{fig:symmetry_closest_appr}. Due to this symmetry, a good approximation to the integral of Equation \eqref{eq:R_defl} is given by

\begin{equation}
    \Bar{R}_{ij} = \int_0^{\Bar{b}'} R_{ij}(b; \Bar{g}_{ij}) \d b
\end{equation}
with $\Bar{b}'$ given by
\begin{equation}
    \theta_{ij}(\Bar{b}'; \Bar{g}_{ij}) = 0.
\end{equation}

\begin{figure}[htb]
    \centering
    \includegraphics[width=\textwidth]{symmetry_closest_appr.png}
    \caption{The distance of closest approach as a function of dimensionless relative velocity $g$ and impact parameter $b$ at $T = \SI{500}{\kelvin}$}
    \label{fig:symmetry_closest_appr}
\end{figure}

This integral was evaluated using a six-point Gauss-Legendre quadrature, after investigating the convergence behaviour of the quadrature and finding that this was sufficient to achieve a relative precision of $\approx 10^{-8}$.

\bibliographystyle{ieeetr}
\bibliography{bibliografi}

\end{document}

\section{Conductivity}
To compute the conductivity, we must first find the Sonine polynomial expansion coefficients related to the temperature response function of component $i$, $\Vec{A}_i$, denoted $a_q^{(i)}$. To do this solve the set of equations 
\begin{equation}
    \sum_{j = 1}^s \sum_{q = 0}^{N} \Lambda_{ij}^{pq} a_q^{(j)} = \frac{4}{5k_B}x_i K_i \delta_{p,1}, \hspace{1cm} s = \{1, 2, ..., s\}, p = \{0, 1, ..., N\}
\end{equation}
where $s$ is the number of components, $N$ is the order of the Enskog approximation and $K_i$ is given by
\begin{equation}
    K_i = 1 + \frac{12}{5}\sum_{j = 1}^s \rho b_{ij} M_{ij} M_{ji} g_{ij}
\end{equation}
where $M_{ij} = \frac{m_i}{m_i + m_j}$, $\rho b_{ij} = \frac{2 \pi}{3} n_j \sigma_{ij}^3$ and $g_{ij}$ is the radial distribution function at contact. $\sigma_{ij}$ is taken to be the contact distance of the rdf.

Solve this set of equations by writing it in matrix form as 
\begin{equation}
    \Mat{\Lambda} \Vec{a} = \pmb{\lambda}
\end{equation}
with
\begin{equation}
    \Mat{\Lambda} = \left.\overbrace{
    \begin{bmatrix}
\Lambda_{11}^{00} & \Lambda_{12}^{00} & \hdots & \Lambda_{1s}^{00} & \Lambda_{11}^{01} & \hdots & \hdots & \Lambda_{1s}^{01} & \hdots & \hdots & \Lambda_{1s}^{0N} \\
\Lambda_{21}^{00} & \ddots & & \vdots & \vdots & \ddots & & \vdots & \ddots & & \vdots \\
\vdots & & \ddots & \vdots & \vdots & & \ddots & \vdots & & \ddots & \vdots \\
\Lambda_{s1}^{00} & \Lambda_{s2}^{00} & \hdots & \Lambda_{ss}^{00} & \Lambda_{s1}^{01} & \hdots & \hdots & \Lambda_{ss}^{01} & \hdots & \hdots & \Lambda_{ss}^{0N} \\
\Lambda_{11}^{10} & \Lambda_{12}^{10} & \hdots & \Lambda_{1s}^{10} & \Lambda_{11}^{11} & \hdots & \hdots & \Lambda_{1s}^{11} & \hdots & \hdots & \Lambda_{1s}^{1N} \\
\vdots & \ddots & & \vdots & \vdots & \ddots & & \vdots & \ddots & & \vdots \\
\vdots & & \ddots & \vdots & \vdots & & \ddots & \vdots & & \ddots & \vdots \\
\Lambda_{s1}^{10} & \hdots & \hdots & \Lambda_{ss}^{10} & \Lambda_{s1}^{11} & \hdots & \hdots & \Lambda_{ss}^{11} & \hdots & \hdots & \Lambda_{ss}^{1N} \\
\vdots & \ddots & & \vdots & \vdots & \ddots & & \vdots & \ddots & & \vdots \\
\vdots & & \ddots & \vdots & \vdots & & \ddots & \vdots & & \ddots & \vdots \\
\Lambda_{s1}^{N0} & \hdots & \hdots & \Lambda_{ss}^{N0} & \Lambda_{s1}^{N1} & \hdots & \hdots & \Lambda_{ss}^{N1} & \hdots & \hdots & \Lambda_{ss}^{NN} \\
    \end{bmatrix}
    }^{Ns} \right\} Ns
\end{equation}

which for clarity may be written as an ($N \times N$) block matrix
\begin{equation}
    \Mat{\Lambda} = 
    \begin{bmatrix}
        \Mat{\Lambda}^{(00)} & \Mat{\Lambda}^{(01)} & \hdots & \Mat{\Lambda}^{(0N)} \\
        \Mat{\Lambda}^{(10)} & \ddots             &        & \vdots \\
        \vdots             &                    & \ddots & \vdots \\
        \Mat{\Lambda}^{(N0)} & \hdots & \hdots  & \Mat{\Lambda}^{(NN)}
    \end{bmatrix}
    \label{eq:lambda_block1}
\end{equation}
consisting of ($s \times s$) blocks
\begin{equation}
    \Mat{\Lambda}^{(pq)} = 
    \begin{bmatrix}
        \Lambda_{11}^{pq} & \Lambda_{12}^{pq} & \hdots & \Lambda_{1s}^{pq} \\
        \Lambda_{21}^{pq} & \ddots & & \vdots \\
        \vdots & & \ddots & \vdots \\
        \Lambda_{s1}^{pq} & \hdots & \hdots & \Lambda_{ss}^{pq} 
    \end{bmatrix}.
    \label{eq:lambda_block2}
\end{equation}

The corresponding vectors $\Vec{a}$ and $\Vec{\delta}$ are then

\begin{equation}
    \Vec{a} = 
    \begin{pmatrix}
        a_0^{(1)} \\ a_0^{(2)} \\ \vdots \\ a_0^{(s)} \\ a_1^{(1)} \\ \vdots \\ a_{1}^{(s)} \\ \vdots \\ a_{N}^{(s)}
    \end{pmatrix}
    , \hspace{2cm}
    \pmb{\lambda} = \frac{4}{5k_B}
    \begin{pmatrix}
        0 \\ \vdots \\ \times s \\ \vdots \\ 0 \\ x_1 K_1 \\ x_2 K_2 \\ \vdots \\ x_s K_s \\ 0 \\ \vdots \\ \times (N - 2)s \\ \vdots \\ 0
    \end{pmatrix}.
\end{equation}

The $\Lambda_{ij}^{pq}$ elements are evaluated as
\begin{equation}
    \Lambda_{ij}^{pq} = \frac{8}{75 k_B^2 T} \sqrt{m_1 m_2} \left\{ x_i x_j g_{ij} \left[S_{3/2}^{(p)}(\vsU_i^2)\vsU_i, S_{3/2}^{(q)}(\vsU_j^2)\vsU_j\right]_{ij} + \delta_{i,j} \sum_{\ell = 1}^{s} x_i x_\ell g_{i\ell} \left[S_{3/2}^{(p)}(\vsU_i^2)\vsU_i, S_{3/2}^{(q)}(\vsU_i^2)\vsU_i\right]_{i\ell} \right\}
    \label{eq:lambda_coeffs}
\end{equation}
where $[F_i, G_j]_{ij}$ and $[F_i, G_i]_{ij}$ refer to the bracket integrals to which Thompson, Tipton and Lloyalka have identified summational expressions for and $\delta_{i,j}$ is the Kronecker delta.

The conductivity is given in terms of the expansion coefficients $\Vec{a}_q^{(j)}$ as 
\begin{equation}
    \begin{split}
        \lambda &= \frac{5 k_B}{4} \sum_{i} x_i \left(1 + \frac{12}{5} \sum_{j} \rho b_{ij} M_{ij} M_{ji} g_{ij} \right)\left(a_1^{(i)} - \sum_{k} d_{i,1}^{(k)}d_k^{th}\right) \\
        & \hspace{.5cm} +\frac{5 k_B}{4} \sum_i \sum_j \sqrt{\frac{2\pi m_i m_j k_B T}{m_i + m_j}} \frac{n_i n_j}{m_i + m_j}\sigma_{ij}^{4} g_{ij}
    \end{split}
\end{equation}
where the sums run over the components of the mixture. The expression may be simplified somewhat by inserting for $K_i$ to yield
\begin{equation}
    \begin{split}
        \lambda &= \frac{5 k_B}{4} \sum_{i} x_i K_i \left(a_1^{(i)} - \sum_{k} d_{i,1}^{(k)}d_k^{th}\right) \\
        & \hspace{.5cm} +\frac{5 k_B}{4} \sum_i \sum_j \sqrt{\frac{2\pi m_i m_j k_B T}{m_i + m_j}} \frac{n_i n_j}{m_i + m_j}\sigma_{ij}^{4} g_{ij}.
    \end{split}
\end{equation}
The coefficients $d_k^{th}$ are found by the solving the set of equations 
\begin{equation}
    \sum_{k = 1}^s d_{i,0}^{(k)} d_k^{th} = a_0^{(i)}, \hspace{1cm} i = \{1, 2, ..., s\},
    \label{eq:dth_set}
\end{equation}
with $d_{i,0}^{(k)}$ given by the solution to the diffusion equations \eqref{eq:diffusion_eq_matr}. Written in matrix form, equation \eqref{eq:dth_set} reads
\begin{equation}
    \Mat{D}_{th} \Vec{d}_{th} = \Vec{a}_0
\end{equation}
with 
\begin{equation}
    \Mat{D}_{th} = 
    \begin{bmatrix}
        d_{1, 0}^{(1)} & d_{1, 0}^{(2)} & \hdots & d_{1, 0}^{(s)} \\
        d_{2, 0}^{(1)} & \ddots & & \vdots \\
        \vdots & & \ddots & \vdots \\
        d_{s, 0}^{(1)} & d_{s, 0}^{(2)} & \hdots & d_{s, 0}^{(s)}
    \end{bmatrix}
    , \hspace{1cm}
    \Vec{a_0} = 
    \begin{pmatrix}
        a_0^{(1)} \\ a_0^{(2)} \\ \vdots \\ a_0^{(s)}
    \end{pmatrix}
\end{equation}




