\section{Diffusion}

The molar flux of species $i$ in the centre-of-mass frame of reference is related to the vdf. as
\begin{equation}
    \mflux_i = n_i (\bvu_i - \vu^m) = \int f_i \vU_i \d \vu_i.
    \label{eq:molar_flux_mass_for}
\end{equation}
The diffusive response functions $\vD_i^{(j)}$ are written as the Sonine polynomial expansions
\begin{equation}
    \vD_i^{(j)} = \frac{m_i}{2 k_B T} \sum_{p = 0}^\infty d_{i, j}^{(p)} S_{3/2}^{(p)}(\sU^2)
\end{equation}
At constant temperature, the integral of Equation \eqref{eq:molar_flux_mass_for} may be evaluated in terms of the $d_{i,j}^{(r)}$ expansion coefficients as
\begin{equation}
    \mflux_i = - \frac{x_i}{2} \sum_j d_{i, j}^{(0)} \Vec{d}_j
\end{equation}
where $\Vec{d}_j$ may be appropriately simplified in the absence of a temperature gradient.

This section will first describe the equations that must be solved to determine the expansion coefficients of the diffusive response function, then the relationship between the diffusive fluxes and driving forces is discussed. Finally, several ways of defining the diffusion coefficient in a mixture are introduced, and explicit expressions relating the diffusion coefficients to the expansion coefficients are given.

\subsection{Determining the expansion coefficients}
\label{sec:diffusion_exp_coeff}

Using the orthogonality properties of the Sonine polynomials and inserting this expansion into the constrains posed by the summational invariants, one finds that the expansion coefficients must satisfy 
\begin{equation}
    \begin{split}
        \sum_{j=1}^s \sum_{q = 0}^{N} \Lambda_{ij}^{pq}d_{j,k}^{(q)} &= \frac{8}{25 k_B}\left(\delta_{i,k} - \frac{\rho_i}{\rho}\right)\delta_{p,0}, \hspace{1cm}
        \begin{cases}
            i = \{2, 3, ..., s\}&\\
            p = \{0, 1, ..., N\}& \\
            k = \{1, 2, ..., s\}&
        \end{cases}
        \\
        \sum_{j=1}^s \sum_{q=0}^N \Lambda_{1j}^{(pq)} d_{j,k}^{(q)} &= 0, \hspace{1cm}
        \begin{cases}
            k = \{1, 2, ..., s\}&\\
            p = \{1, 2, ..., s\}&
        \end{cases}
        \\
        \sum_{j = 1}^s \frac{\rho_j}{\rho} d_{j,k}^{(0)} &= 0, \hspace{1cm} k = \{1, 2, ..., s\}
    \end{split}
    \label{eq:diffusion_eq_set}
\end{equation}

where $\Lambda_{ij}^{pq}$ are given by

\begin{equation}
    \Lambda_{ij}^{pq} = x_i x_j \chi_{ij} \lsp S_{3/2}^{(p)}(\sU_i^2)\vsU_i^2, S_{3/2}^{(p)}(\sU_j^2)\vsU_j^{2}\rsp_{ij} + \delta_{i, j} x_i \sum_{k} x_{k} \chi_{ik} \lsp S_{3/2}^{(p)}(\sU_i^2)\vsU_i^2, S_{3/2}^{(p)}(\sU_i^2)\vsU_i^{2}\rsp_{ik}.
    \label{eq:lambda_ijpq}
\end{equation}

The square bracket integrals may be written as linear combinations of the collision integrals.\cite{kinetics_1} Thompson et al. identify these integrals as

\begin{equation}
    \begin{split}
        \lsp S_{3/2}^{(p)}(\sU_i^2)\vsU_i^2, S_{3/2}^{(p)}(\sU_j^2)\vsU_j^{2}\rsp_{ij} &= 8 M_{ij}^{q + \frac{1}{2}} M_{ji}^{p + \frac{1}{2}} \sum_{\ell = 1}^{\min [p, q] + 1} \sum_{r = \ell}^{p + q + 2 - \ell} A_{pqr \ell} \Omega_{ij}^{(\ell)}(r) \\
        \lsp S_{3/2}^{(p)}(\sU_i^2)\vsU_i^2, S_{3/2}^{(p)}(\sU_i^2)\vsU_i^{2}\rsp_{ik} &= 8 \sum_{\ell = 1}^{\min [p, q] + 1} \sum_{r = \ell}^{p + q + 2 - \ell} A_{pqr \ell}' \Omega_{ik}^{(\ell)}(r).
    \end{split}
\end{equation}

The set of equations \eqref{eq:diffusion_eq_set} may be written as

\begin{equation}
    \Mat{D} \Vec{d} = \Vec{\delta}
    \label{eq:diffusion_eq_matr}
\end{equation}
where $\Mat{D}$ is a ($Ns^2 \times Ns^2$) matrix consisting of the blocks
\begin{equation}
    \Mat{D} = 
    \begin{bmatrix}
        \Mat{\omega} \\ \Mat{\Lambda}_1^{(p>0)} \\ \Mat{\Lambda}_{i>1}
    \end{bmatrix}
\end{equation}

with
\begin{equation}
    \begin{split}
        \Mat{\omega} &= 
        \begin{bmatrix}
            \Vec{\omega}^{(1)} \\ \Vec{\omega}^{(2)} \\ \vdots \\ \Vec{\omega}^{(s)}
        \end{bmatrix}\\
        \Vec{\omega}^{(k)} &= 
        \begin{pmatrix}
            0 & \hdots & \times Ns(k - 1) & \hdots & 0 & \omega_1 & \omega_2 & \hdots & \omega_s & 0 & \hdots & \times s(Ns - N(k - 1) - 1) & \hdots & 0
        \end{pmatrix}
    \end{split}
\end{equation}

\begin{equation}
    \begin{split}
        \Mat{\Lambda}_1^{(p>0)} &= 
        \begin{bmatrix}
            \Vec{\Lambda}(1, 1) \\ \Vec{\Lambda}(1, 2) \\ \vdots \\ \Vec{\Lambda}(1, N) \\ \Vec{\Lambda}(2, 1) \\ \vdots \\ \Vec{\Lambda}(s, N)
        \end{bmatrix},\\
        \Vec{\Lambda}(k, p) &= 
        \begin{pmatrix}
            0 & \hdots & \times Ns(k - 1) & \hdots & 0 & \Lambda_{11}^{p0} & \Lambda_{12}^{p0} & \hdots & \Lambda_{1s}^{p0} & & \Lambda_{11}^{p1} & \hdots & \Lambda_{1s}^{pN} & 0 & \hdots & \times Ns(s - k) & \hdots & 0\\
        \end{pmatrix}
    \end{split}
\end{equation}

\begin{equation}
    \begin{split}
        \Mat{\Lambda}_{i>1} &= 
        \begin{bmatrix}
            \Mat{\Lambda}_{k=1} & & & \\
            & \Mat{\Lambda}_{k=2} & & \\
            & & \ddots & \\
            & & & \Mat{\Lambda}_{k=s}    
        \end{bmatrix}
        \\
        \Mat{\Lambda}_{k} = 
        \begin{bmatrix}
            \Mat{\Lambda}_{i>1}^{(00)} & \Mat{\Lambda}_{i>1}^{(01)} & \hdots & \Mat{\Lambda}_{i>1}^{(0N)} \\
            \Mat{\Lambda}_{i>1}^{(10)} & \ddots             &        & \vdots \\
            \vdots             &                    & \ddots & \vdots \\
            \Mat{\Lambda}_{i>1}^{(N0)} & \hdots & \hdots  & \Mat{\Lambda}_{i>1}^{(NN)}
        \end{bmatrix}&, \hspace{.5cm}
        \Mat{\Lambda}_{i>0}^{(pq)} = 
        \begin{bmatrix}
            \Lambda_{21}^{pq} & \Lambda_{22}^{pq} & \hdots & \hdots & \Lambda_{2s}^{pq} \\
            \Lambda_{31}^{pq} & \ddots & & & \vdots \\
            \vdots & & \ddots & & \vdots \\
            \Lambda_{s1}^{pq} & \hdots & \hdots & \Lambda_{s,(s-1)}^{pq} & \Lambda_{ss}^{pq}
        \end{bmatrix}
    \end{split}
\end{equation}

$\Vec{d}$ and $\Vec{\delta}$ are given as

\begin{equation}
    \Vec{d} = 
    \begin{pmatrix}
        d_{1, 1}^{(0)} \\ d_{2, 1}^{(0)} \\ \vdots \\ d_{s, 1}^{(0)} \\ d_{1, 1}^{(1)} \\ \vdots \\ d_{s, 1}^{(1)} \\ \vdots \\ d_{s, 1}^{(N)} \\ d_{1, 2}^{(0)} \\ \vdots \\ d_{s, 2}^{(N)} \\ \vdots \\ d_{s, s}^{(N)}
    \end{pmatrix}
    , \hspace{1cm}
    \Vec{\delta} = - \frac{8}{25 k_B}
    \begin{pmatrix}
        0 \\ \vdots \\ \times Ns \\ \vdots \\ 0 \\ \Vec{\delta}^{(1)} \\ \Vec{\delta}^{(2)} \\ \vdots \\ \Vec{\delta}^{(s)}
    \end{pmatrix}, \hspace{1cm}
    \Vec{\delta}^{(k)} = 
    \begin{pmatrix}
        \omega_2 \\ \vdots \\ \omega_{k-1} \\ \omega_k - 1 \\ \omega_{k+1} \\ \vdots \\ \omega_s \\ 0 \\ \vdots \\ \times (N-1)(s-1) \\ \vdots \\ 0
    \end{pmatrix}
\end{equation}

\subsection{Flux-Force relations}

Inserting the polynomial expansion of $\vD_i^{(j)}$ into Equation \eqref{eq:molar_flux_mass_for} one finds that at uniform temperature and pressure, and in the absence of external forces
\begin{equation}
    \begin{split}
        \mflux_i &= - \frac{x_i}{2} \sum_j d_{i,j}^{(0)} \Vec{d}_j\\
        &= - \frac{x_i}{2n} \sum_j d_{i,j}^{(0)} \sum_k E_{jk} \nabla n_k \\
        &\equiv - \frac{x_i}{2n}\sum_k \nabla n_k \sum_j E_{jk} d_{i,j}^{(0)}
    \end{split}
    \label{eq:diffusive_flux}
\end{equation}

where $E_{ij} \equiv \frac{n_i}{k_B T} \ppder{\mu_i}{n_j}_{T,n_{k \neq j}}$.

At this point it should be noted that the driving forces $\Vec{d}_j$ are not independent, but satisfy the relation $\sum_j \Vec{d}_j = 0$, following from the Gibbs-Duhem equation. Due to this dependency there are several manners in which one may identify the diffusion coefficient of a multicomponent mixture. Equation \eqref{eq:diffusive_flux} will be the starting point from which the diffusion coefficient is related to the polynomial expansion coefficients. It is therefore convenient to denote

\begin{equation}
    \mflux_i = - \sum_j D^{(E,m)}_{ij} \nabla n_j
    \label{eq:diffusive_flux2}
\end{equation}
where 
\begin{equation}
    D^{(E,m)}_{ij} \equiv \frac{x_i}{2n} \sum_k E_{kj} d_{i,k}^{(0)}
\end{equation}

In order to identify the diffusion coefficient related to different driving forces, it may be useful to express $\mflux_i$ in terms of the mole fraction gradients $\nabla x_i$, as these conveniently sum to zero. Taking

\begin{equation}
    \begin{split}
        \nabla n_k &= n \nabla x_k + x_k \nabla n\\
        &= n\nabla x_k + x_k \sum_\ell \ppder{n}{x_\ell}_{T,p} \nabla x_\ell \\
        &= n \nabla x_k + x_k \sum_\ell - \frac{\ppder{p}{x_\ell}_{T,n}}{\ppder{p}{n}_{T,\vx}} \nabla x_\ell \\
        &= n \nabla x_k + x_k \sum_\ell - \frac{\ppder{T}{x_\ell}_{p,n}}{\ppder{T}{n}_{p,\vx}} \nabla x_\ell \\
        &= \sum_{\ell} \lsp n \delta_{k,\ell} - x_k \frac{\ppder{p}{x_\ell}_{T,n}}{\ppder{p}{n}_{T,\vx}} \rsp \nabla x_\ell \\
        &\equiv \sum_{\ell} \vartheta_{k\ell} \nabla x_\ell
    \end{split}
    \label{eq:nabla_k_nabla_x}
\end{equation}

where the final equality defines $\vartheta_{k \ell}$ and the equality on the fourth line can be derived by differentiating the total differential of density

\begin{equation}
    \d n = \ppder{n}{T}_{p, \vx} \dT + \ppder{n}{p}_{T, \vx} \dT + \sum_i \ppder{n}{x}_{T, p} \d x_i
\end{equation}

with respect to $x_i$ and $n$, and using the fact that $\ppder{n}{x_i}_{p, n} = \ppder{n}{x_i}_{T, n} = 0$ and $\ppder{n}{n}_{p, \vx} = \ppder{n}{n}_{T, \vx} = 1$. Inserting Equation \eqref{eq:nabla_k_nabla_x} into Equation \eqref{eq:diffusive_flux} yields
\begin{equation}
    \begin{split}
        \mflux_i &= - \frac{x_i}{2n}\sum_k \sum_{\ell} \vartheta_{k \ell} \nabla x_\ell \sum_j E_{jk} d_{ij}^{(0)}\\
        &= - \frac{x_i}{2n} \sum_\ell \nabla x_\ell \sum_{k} \vartheta_{k \ell} \sum_{j} E_{jk} d_{ij}^{(0)}.
    \end{split}
\end{equation}

Alternatively, one may choose to express the flux using a set of independent gradients. Using the condition $\sum_j \Vec{d}_j = 0$ to express $\nabla n_i$ as 

\begin{equation}
    \begin{split}
        \sum_j \Vec{d}_j = \sum_j \sum_k E_{jk}\nabla n_k &= 0\\
        \nabla n_i \sum_j E_{ji} + \sum_{k \neq i} \nabla n_k \sum_j E_{jk} &= 0\\
        \nabla n_i = - \sum_{k \neq i} \nabla n_k \frac{\sum_j E_{jk}}{\sum_j E_{ji}} \equiv - \sum_{k \neq i} \nabla n_k \frac{E'_k}{E'_i},
    \end{split}
    \label{eq:molar_grad_relation}
\end{equation}

where the final equality defines $E'_i = \sum_{j} E_{ji}$, and inserting into Equation \eqref{eq:diffusive_flux} one arrives at

\begin{equation}
    \begin{split}
        \mflux_i &= - \frac{x_i}{2n}\sum_k \nabla n_k \sum_j E_{jk} d_{ij}^{(0)} \\
        &= - \frac{x_i}{2n} \lrp \nabla n_i \sum_j E_{ji} d_{ij}^{(0)} + \sum_{k \neq i} \nabla n_k \sum_j E_{jk} d_{ij}^{(0)} \rrp \\
        &= - \frac{x_i}{2n}\sum_{k \neq i} \nabla n_k \sum_j (E_{jk} - E_{ji} \frac{E'_k}{E'_i}) d_{ij}^{(0)}.
    \end{split}
    \label{eq:diff_fluxforce_3}
\end{equation}

As shown elsewhere,\cite{prosjektoppgave} the molar fluxes in the centre of mass FoR can be translated to another frame of reference $B$ via the matrix transformation

\begin{equation}
    \Vec{J}^{(B)} = \Mat{\Psi}^{B,m} \Vec{J}^{(m)}
    \label{eq:flux_transform_for}
\end{equation}
where 
\begin{equation}
    \psi_{ij}^{Bm} = \delta_{i,j} - x_i \left( \frac{b_j}{b} - \frac{M_j b_\ell}{b M_\ell}\right)
\end{equation}

where $\delta_{i,j}$ is the Kronecker delta, $x_i$ denotes the mole fraction of species $i$, $M_j$ is the molar mass of species $j$, $b = \ppder{B}{n}_{T,p,\Vec{x}}$ is the molar value of the extensive property $B$, $b_j \equiv \ppder{B}{n_j}_{T,p,n_{k \neq j}}$ is the partial molar value of $B$ with respect to species $j$ and $\ell$ denotes an arbitrary component.

Note that because $\ell$ denotes an arbitrary component

The vector $\Vec{J}^{(B)}$ is
\begin{equation}
    J^{(B)} = 
    \begin{pmatrix}
        J^B_1 \\ J^B_2 \\ \vdots \\ J^B_s
    \end{pmatrix}
\end{equation}
and equivalent for $\Vec{J}^{(m)}$. 

This transformation can also be used to translate the ''apparent'' diffusion coefficients between different frames of reference, as is discussed in the following section.

\subsection{Defining the diffusion coefficients}
The diffusion coefficients of a mixture may be defined through the relationship between the fluxes and driving forces in a mixture. In this section Equation \eqref{eq:diffusive_flux2}, describing the relationship between the molar flux in the centre of mass FoR and the molar density gradients, will be related to other flux-force relationships through which the diffusion coefficients are commonly defined.

\subsubsection{Binary mixtures}

The molar flux in a binary mixture is often described by Ficks law in the centre-of-moles frame of reference. This is convenient in the case of fairly dilute mixtures in which equimolar counter diffusion applies. Ficks law for a binary mixture reads
\begin{equation}
    \flux{1}{n,n} = - D_{12}^{\text{Fick}}\nabla n_1
\end{equation}
where the superscript $(n, n)$ indicates that the flux is in the molar basis, in the centre-of-moles FoR. Translating the fluxes of Equation \eqref{eq:diffusive_flux2} to the centre of moles frame of reference yields
\begin{equation}
    \begin{pmatrix} \flux{1}{n,n} \\ \flux{1}{n,n} \end{pmatrix}
    =
    - \Mat{\Psi}^{(n,m)}\Mat{D}^{(E,m)} \begin{pmatrix} \nabla n_1 \\ \nabla n_2 \end{pmatrix}
    \equiv
    - \Mat{D}^{(E,n)} \begin{pmatrix} \nabla n_1 \\ \nabla n_2 \end{pmatrix}.
\end{equation}

where the final equality defines $\Mat{D}^{(E,n)} \equiv \Mat{\Psi}^{(n,m)}\Mat{D}^{(E,m)}$. By Equation \eqref{eq:molar_grad_relation}
\begin{equation}
    \nabla n_2 = - \frac{E'_1}{E'_2} \nabla n_1 
\end{equation}
such that 
\begin{equation}
    \flux{1}{n,n} = - D_{11}^{(E,n)} \nabla n_1 - D_{12}^{(E,n)} \nabla n_2 = \lrp D_{12}^{(E,n)} \frac{E'_1}{E'_2} - D_{11}^{(E,n)} \rrp \nabla n_1.
\end{equation}
Thereby, the Fickean diffusion coefficient in the molar basis is identified as
\begin{equation}
    D_{12}^{\text{Fick}} = D_{12}^{(E,n)} \frac{E'_1}{E'_2} - D_{11}^{(E,n)}.
\end{equation}

\subsubsection{Multicomponent mixtures}

de Haro et al. define the diffusion coefficient through a multicomponent generalization of Ficks law in the centre of mass FoR
\begin{equation}
    \mflux_i = \frac{1}{m_i} \sum_{j \neq i} D_{ij}^{fick} m_j \nabla n_j.
    \label{eq:multicomp_fick}
\end{equation}
This expression has the advantage of reducing to the commonly used expression 
\begin{equation}
    \mflux_i = - D_{12} \nabla n_1
\end{equation}

in the case of a binary mixture. Comparing Equation \eqref{eq:multicomp_fick} to \eqref{eq:diff_fluxforce_3} one finds that
\begin{equation}
    D_{ij}^{fick} = - \frac{x_i m_i }{2n m_j} \sum_j (E_{jk} - E_{ji} \frac{E'_k}{E'_i}) d_{ij}^{(0)}.
    \label{eq:fick_diffusion_coeff}
\end{equation}

An alternative to the generalized Fickean diffusion coefficient is the Maxwell-Stefan diffusion coefficient for multicomponent mixtures. It is defined by
\begin{equation}
    \nabla x_i = - \sum_{j \neq i} \frac{x_i x_j}{D_{ij}^{M.S.}} (\bvu_i - \bvu_j)
\end{equation}

As shown by 

For a realistic mixture, it is of interest to separate the factors $E_{ij}$, which may be computed from an equation of state (EoS) from the polynomial expansion coefficients $d_{i,j}^{(0)}$ which can be computed from kinetic theory without requiring an accurate equation of state. Writing the fluxes as

\begin{equation}
    \mflux = \Mat{D}^{kin} \Mat{\Gamma} \nabla \vn
\end{equation}

and comparing to Equation \eqref{eq:diffusive_flux}, one finds that this separation can be achieved by using

\begin{equation}
    \Mat{D}^{kin} = - \frac{1}{2n}
    \begin{bmatrix}
        x_1 d_{1,1}^{(0)} & x_2 d_{1,2}^{(0)} & \hdots & x_1 d_{1,s}^{(0)} \\
        x_1 d_{2,1}^{(0)} & x_2 d_{2,2}^{(0)} & & \vdots \\
        \vdots & \vdots & \ddots & \vdots \\
        x_1 d_{s,1}^{(0)} & x_2 d_{s,2}^{(0)} & \hdots & x_s d_{s,s}^{(0)}
    \end{bmatrix}
    , \hspace{.5cm}
    \Mat{\Gamma} = 
    \begin{bmatrix}
        E_{11} & E_{12} & \hdots & E_{1s} \\
        E_{21} & E_{22} & & \vdots \\
        \vdots & \vdots & \ddots & \vdots \\
        E_{s1} & E_{s2} & \hdots & E_{ss}
    \end{bmatrix}
\end{equation}

One should note that these matrices are not invertible and that the coefficients do not reduce to the Fickean diffusion coefficients in the case of a binary mixture. This formulation is still convenient, as it allows one to transform the coefficients directly to another frame of reference $B$ by the transformation

\begin{equation}
    \begin{split}
        \flux{}{(B)} &= \Mat{\Psi}^{B, m} \mflux \\
        &= \Mat{\Psi}^{B, m} \Mat{D}^{kin} \Mat{\Gamma} \nabla \vn \\
        &= \Mat{D}^{kin, B} \Mat{\Gamma} \nabla \vn \\
        \Mat{D}^{kin, B} &\equiv \Mat{\Psi}^{B, m} \Mat{D}^{kin}
    \end{split}
\end{equation}

where $\Mat{D}^{kin, B}$ is the apparent kinetic diffusion coefficient matrix in the $B$ FoR. This may then be transformed as desired by the same procedure as that used to arrive at Equation \eqref{eq:fick_diffusion_coeff} of one wishes to express the fluxes as a function of only independent gradients.